
% Default to the notebook output style

    


% Inherit from the specified cell style.




    
\documentclass[11pt]{article}

    
    
    \usepackage[T1]{fontenc}
    % Nicer default font (+ math font) than Computer Modern for most use cases
    \usepackage{mathpazo}

    % Basic figure setup, for now with no caption control since it's done
    % automatically by Pandoc (which extracts ![](path) syntax from Markdown).
    \usepackage{graphicx}
    % We will generate all images so they have a width \maxwidth. This means
    % that they will get their normal width if they fit onto the page, but
    % are scaled down if they would overflow the margins.
    \makeatletter
    \def\maxwidth{\ifdim\Gin@nat@width>\linewidth\linewidth
    \else\Gin@nat@width\fi}
    \makeatother
    \let\Oldincludegraphics\includegraphics
    % Set max figure width to be 80% of text width, for now hardcoded.
    \renewcommand{\includegraphics}[1]{\Oldincludegraphics[width=.8\maxwidth]{#1}}
    % Ensure that by default, figures have no caption (until we provide a
    % proper Figure object with a Caption API and a way to capture that
    % in the conversion process - todo).
    \usepackage{caption}
    \DeclareCaptionLabelFormat{nolabel}{}
    \captionsetup{labelformat=nolabel}

    \usepackage{adjustbox} % Used to constrain images to a maximum size 
    \usepackage{xcolor} % Allow colors to be defined
    \usepackage{enumerate} % Needed for markdown enumerations to work
    \usepackage{geometry} % Used to adjust the document margins
    \usepackage{amsmath} % Equations
    \usepackage{amssymb} % Equations
    \usepackage{textcomp} % defines textquotesingle
    % Hack from http://tex.stackexchange.com/a/47451/13684:
    \AtBeginDocument{%
        \def\PYZsq{\textquotesingle}% Upright quotes in Pygmentized code
    }
    \usepackage{upquote} % Upright quotes for verbatim code
    \usepackage{eurosym} % defines \euro
    \usepackage[mathletters]{ucs} % Extended unicode (utf-8) support
    \usepackage[utf8x]{inputenc} % Allow utf-8 characters in the tex document
    \usepackage{fancyvrb} % verbatim replacement that allows latex
    \usepackage{grffile} % extends the file name processing of package graphics 
                         % to support a larger range 
    % The hyperref package gives us a pdf with properly built
    % internal navigation ('pdf bookmarks' for the table of contents,
    % internal cross-reference links, web links for URLs, etc.)
    \usepackage{hyperref}
    \usepackage{longtable} % longtable support required by pandoc >1.10
    \usepackage{booktabs}  % table support for pandoc > 1.12.2
    \usepackage[inline]{enumitem} % IRkernel/repr support (it uses the enumerate* environment)
    \usepackage[normalem]{ulem} % ulem is needed to support strikethroughs (\sout)
                                % normalem makes italics be italics, not underlines
    

    
    
    % Colors for the hyperref package
    \definecolor{urlcolor}{rgb}{0,.145,.698}
    \definecolor{linkcolor}{rgb}{.71,0.21,0.01}
    \definecolor{citecolor}{rgb}{.12,.54,.11}

    % ANSI colors
    \definecolor{ansi-black}{HTML}{3E424D}
    \definecolor{ansi-black-intense}{HTML}{282C36}
    \definecolor{ansi-red}{HTML}{E75C58}
    \definecolor{ansi-red-intense}{HTML}{B22B31}
    \definecolor{ansi-green}{HTML}{00A250}
    \definecolor{ansi-green-intense}{HTML}{007427}
    \definecolor{ansi-yellow}{HTML}{DDB62B}
    \definecolor{ansi-yellow-intense}{HTML}{B27D12}
    \definecolor{ansi-blue}{HTML}{208FFB}
    \definecolor{ansi-blue-intense}{HTML}{0065CA}
    \definecolor{ansi-magenta}{HTML}{D160C4}
    \definecolor{ansi-magenta-intense}{HTML}{A03196}
    \definecolor{ansi-cyan}{HTML}{60C6C8}
    \definecolor{ansi-cyan-intense}{HTML}{258F8F}
    \definecolor{ansi-white}{HTML}{C5C1B4}
    \definecolor{ansi-white-intense}{HTML}{A1A6B2}

    % commands and environments needed by pandoc snippets
    % extracted from the output of `pandoc -s`
    \providecommand{\tightlist}{%
      \setlength{\itemsep}{0pt}\setlength{\parskip}{0pt}}
    \DefineVerbatimEnvironment{Highlighting}{Verbatim}{commandchars=\\\{\}}
    % Add ',fontsize=\small' for more characters per line
    \newenvironment{Shaded}{}{}
    \newcommand{\KeywordTok}[1]{\textcolor[rgb]{0.00,0.44,0.13}{\textbf{{#1}}}}
    \newcommand{\DataTypeTok}[1]{\textcolor[rgb]{0.56,0.13,0.00}{{#1}}}
    \newcommand{\DecValTok}[1]{\textcolor[rgb]{0.25,0.63,0.44}{{#1}}}
    \newcommand{\BaseNTok}[1]{\textcolor[rgb]{0.25,0.63,0.44}{{#1}}}
    \newcommand{\FloatTok}[1]{\textcolor[rgb]{0.25,0.63,0.44}{{#1}}}
    \newcommand{\CharTok}[1]{\textcolor[rgb]{0.25,0.44,0.63}{{#1}}}
    \newcommand{\StringTok}[1]{\textcolor[rgb]{0.25,0.44,0.63}{{#1}}}
    \newcommand{\CommentTok}[1]{\textcolor[rgb]{0.38,0.63,0.69}{\textit{{#1}}}}
    \newcommand{\OtherTok}[1]{\textcolor[rgb]{0.00,0.44,0.13}{{#1}}}
    \newcommand{\AlertTok}[1]{\textcolor[rgb]{1.00,0.00,0.00}{\textbf{{#1}}}}
    \newcommand{\FunctionTok}[1]{\textcolor[rgb]{0.02,0.16,0.49}{{#1}}}
    \newcommand{\RegionMarkerTok}[1]{{#1}}
    \newcommand{\ErrorTok}[1]{\textcolor[rgb]{1.00,0.00,0.00}{\textbf{{#1}}}}
    \newcommand{\NormalTok}[1]{{#1}}
    
    % Additional commands for more recent versions of Pandoc
    \newcommand{\ConstantTok}[1]{\textcolor[rgb]{0.53,0.00,0.00}{{#1}}}
    \newcommand{\SpecialCharTok}[1]{\textcolor[rgb]{0.25,0.44,0.63}{{#1}}}
    \newcommand{\VerbatimStringTok}[1]{\textcolor[rgb]{0.25,0.44,0.63}{{#1}}}
    \newcommand{\SpecialStringTok}[1]{\textcolor[rgb]{0.73,0.40,0.53}{{#1}}}
    \newcommand{\ImportTok}[1]{{#1}}
    \newcommand{\DocumentationTok}[1]{\textcolor[rgb]{0.73,0.13,0.13}{\textit{{#1}}}}
    \newcommand{\AnnotationTok}[1]{\textcolor[rgb]{0.38,0.63,0.69}{\textbf{\textit{{#1}}}}}
    \newcommand{\CommentVarTok}[1]{\textcolor[rgb]{0.38,0.63,0.69}{\textbf{\textit{{#1}}}}}
    \newcommand{\VariableTok}[1]{\textcolor[rgb]{0.10,0.09,0.49}{{#1}}}
    \newcommand{\ControlFlowTok}[1]{\textcolor[rgb]{0.00,0.44,0.13}{\textbf{{#1}}}}
    \newcommand{\OperatorTok}[1]{\textcolor[rgb]{0.40,0.40,0.40}{{#1}}}
    \newcommand{\BuiltInTok}[1]{{#1}}
    \newcommand{\ExtensionTok}[1]{{#1}}
    \newcommand{\PreprocessorTok}[1]{\textcolor[rgb]{0.74,0.48,0.00}{{#1}}}
    \newcommand{\AttributeTok}[1]{\textcolor[rgb]{0.49,0.56,0.16}{{#1}}}
    \newcommand{\InformationTok}[1]{\textcolor[rgb]{0.38,0.63,0.69}{\textbf{\textit{{#1}}}}}
    \newcommand{\WarningTok}[1]{\textcolor[rgb]{0.38,0.63,0.69}{\textbf{\textit{{#1}}}}}
    
    
    % Define a nice break command that doesn't care if a line doesn't already
    % exist.
    \def\br{\hspace*{\fill} \\* }
    % Math Jax compatability definitions
    \def\gt{>}
    \def\lt{<}
    % Document parameters
    \title{Part 6 - Saving and Loading Models (Exercises)}
    
    
    

    % Pygments definitions
    
\makeatletter
\def\PY@reset{\let\PY@it=\relax \let\PY@bf=\relax%
    \let\PY@ul=\relax \let\PY@tc=\relax%
    \let\PY@bc=\relax \let\PY@ff=\relax}
\def\PY@tok#1{\csname PY@tok@#1\endcsname}
\def\PY@toks#1+{\ifx\relax#1\empty\else%
    \PY@tok{#1}\expandafter\PY@toks\fi}
\def\PY@do#1{\PY@bc{\PY@tc{\PY@ul{%
    \PY@it{\PY@bf{\PY@ff{#1}}}}}}}
\def\PY#1#2{\PY@reset\PY@toks#1+\relax+\PY@do{#2}}

\expandafter\def\csname PY@tok@w\endcsname{\def\PY@tc##1{\textcolor[rgb]{0.73,0.73,0.73}{##1}}}
\expandafter\def\csname PY@tok@c\endcsname{\let\PY@it=\textit\def\PY@tc##1{\textcolor[rgb]{0.25,0.50,0.50}{##1}}}
\expandafter\def\csname PY@tok@cp\endcsname{\def\PY@tc##1{\textcolor[rgb]{0.74,0.48,0.00}{##1}}}
\expandafter\def\csname PY@tok@k\endcsname{\let\PY@bf=\textbf\def\PY@tc##1{\textcolor[rgb]{0.00,0.50,0.00}{##1}}}
\expandafter\def\csname PY@tok@kp\endcsname{\def\PY@tc##1{\textcolor[rgb]{0.00,0.50,0.00}{##1}}}
\expandafter\def\csname PY@tok@kt\endcsname{\def\PY@tc##1{\textcolor[rgb]{0.69,0.00,0.25}{##1}}}
\expandafter\def\csname PY@tok@o\endcsname{\def\PY@tc##1{\textcolor[rgb]{0.40,0.40,0.40}{##1}}}
\expandafter\def\csname PY@tok@ow\endcsname{\let\PY@bf=\textbf\def\PY@tc##1{\textcolor[rgb]{0.67,0.13,1.00}{##1}}}
\expandafter\def\csname PY@tok@nb\endcsname{\def\PY@tc##1{\textcolor[rgb]{0.00,0.50,0.00}{##1}}}
\expandafter\def\csname PY@tok@nf\endcsname{\def\PY@tc##1{\textcolor[rgb]{0.00,0.00,1.00}{##1}}}
\expandafter\def\csname PY@tok@nc\endcsname{\let\PY@bf=\textbf\def\PY@tc##1{\textcolor[rgb]{0.00,0.00,1.00}{##1}}}
\expandafter\def\csname PY@tok@nn\endcsname{\let\PY@bf=\textbf\def\PY@tc##1{\textcolor[rgb]{0.00,0.00,1.00}{##1}}}
\expandafter\def\csname PY@tok@ne\endcsname{\let\PY@bf=\textbf\def\PY@tc##1{\textcolor[rgb]{0.82,0.25,0.23}{##1}}}
\expandafter\def\csname PY@tok@nv\endcsname{\def\PY@tc##1{\textcolor[rgb]{0.10,0.09,0.49}{##1}}}
\expandafter\def\csname PY@tok@no\endcsname{\def\PY@tc##1{\textcolor[rgb]{0.53,0.00,0.00}{##1}}}
\expandafter\def\csname PY@tok@nl\endcsname{\def\PY@tc##1{\textcolor[rgb]{0.63,0.63,0.00}{##1}}}
\expandafter\def\csname PY@tok@ni\endcsname{\let\PY@bf=\textbf\def\PY@tc##1{\textcolor[rgb]{0.60,0.60,0.60}{##1}}}
\expandafter\def\csname PY@tok@na\endcsname{\def\PY@tc##1{\textcolor[rgb]{0.49,0.56,0.16}{##1}}}
\expandafter\def\csname PY@tok@nt\endcsname{\let\PY@bf=\textbf\def\PY@tc##1{\textcolor[rgb]{0.00,0.50,0.00}{##1}}}
\expandafter\def\csname PY@tok@nd\endcsname{\def\PY@tc##1{\textcolor[rgb]{0.67,0.13,1.00}{##1}}}
\expandafter\def\csname PY@tok@s\endcsname{\def\PY@tc##1{\textcolor[rgb]{0.73,0.13,0.13}{##1}}}
\expandafter\def\csname PY@tok@sd\endcsname{\let\PY@it=\textit\def\PY@tc##1{\textcolor[rgb]{0.73,0.13,0.13}{##1}}}
\expandafter\def\csname PY@tok@si\endcsname{\let\PY@bf=\textbf\def\PY@tc##1{\textcolor[rgb]{0.73,0.40,0.53}{##1}}}
\expandafter\def\csname PY@tok@se\endcsname{\let\PY@bf=\textbf\def\PY@tc##1{\textcolor[rgb]{0.73,0.40,0.13}{##1}}}
\expandafter\def\csname PY@tok@sr\endcsname{\def\PY@tc##1{\textcolor[rgb]{0.73,0.40,0.53}{##1}}}
\expandafter\def\csname PY@tok@ss\endcsname{\def\PY@tc##1{\textcolor[rgb]{0.10,0.09,0.49}{##1}}}
\expandafter\def\csname PY@tok@sx\endcsname{\def\PY@tc##1{\textcolor[rgb]{0.00,0.50,0.00}{##1}}}
\expandafter\def\csname PY@tok@m\endcsname{\def\PY@tc##1{\textcolor[rgb]{0.40,0.40,0.40}{##1}}}
\expandafter\def\csname PY@tok@gh\endcsname{\let\PY@bf=\textbf\def\PY@tc##1{\textcolor[rgb]{0.00,0.00,0.50}{##1}}}
\expandafter\def\csname PY@tok@gu\endcsname{\let\PY@bf=\textbf\def\PY@tc##1{\textcolor[rgb]{0.50,0.00,0.50}{##1}}}
\expandafter\def\csname PY@tok@gd\endcsname{\def\PY@tc##1{\textcolor[rgb]{0.63,0.00,0.00}{##1}}}
\expandafter\def\csname PY@tok@gi\endcsname{\def\PY@tc##1{\textcolor[rgb]{0.00,0.63,0.00}{##1}}}
\expandafter\def\csname PY@tok@gr\endcsname{\def\PY@tc##1{\textcolor[rgb]{1.00,0.00,0.00}{##1}}}
\expandafter\def\csname PY@tok@ge\endcsname{\let\PY@it=\textit}
\expandafter\def\csname PY@tok@gs\endcsname{\let\PY@bf=\textbf}
\expandafter\def\csname PY@tok@gp\endcsname{\let\PY@bf=\textbf\def\PY@tc##1{\textcolor[rgb]{0.00,0.00,0.50}{##1}}}
\expandafter\def\csname PY@tok@go\endcsname{\def\PY@tc##1{\textcolor[rgb]{0.53,0.53,0.53}{##1}}}
\expandafter\def\csname PY@tok@gt\endcsname{\def\PY@tc##1{\textcolor[rgb]{0.00,0.27,0.87}{##1}}}
\expandafter\def\csname PY@tok@err\endcsname{\def\PY@bc##1{\setlength{\fboxsep}{0pt}\fcolorbox[rgb]{1.00,0.00,0.00}{1,1,1}{\strut ##1}}}
\expandafter\def\csname PY@tok@kc\endcsname{\let\PY@bf=\textbf\def\PY@tc##1{\textcolor[rgb]{0.00,0.50,0.00}{##1}}}
\expandafter\def\csname PY@tok@kd\endcsname{\let\PY@bf=\textbf\def\PY@tc##1{\textcolor[rgb]{0.00,0.50,0.00}{##1}}}
\expandafter\def\csname PY@tok@kn\endcsname{\let\PY@bf=\textbf\def\PY@tc##1{\textcolor[rgb]{0.00,0.50,0.00}{##1}}}
\expandafter\def\csname PY@tok@kr\endcsname{\let\PY@bf=\textbf\def\PY@tc##1{\textcolor[rgb]{0.00,0.50,0.00}{##1}}}
\expandafter\def\csname PY@tok@bp\endcsname{\def\PY@tc##1{\textcolor[rgb]{0.00,0.50,0.00}{##1}}}
\expandafter\def\csname PY@tok@fm\endcsname{\def\PY@tc##1{\textcolor[rgb]{0.00,0.00,1.00}{##1}}}
\expandafter\def\csname PY@tok@vc\endcsname{\def\PY@tc##1{\textcolor[rgb]{0.10,0.09,0.49}{##1}}}
\expandafter\def\csname PY@tok@vg\endcsname{\def\PY@tc##1{\textcolor[rgb]{0.10,0.09,0.49}{##1}}}
\expandafter\def\csname PY@tok@vi\endcsname{\def\PY@tc##1{\textcolor[rgb]{0.10,0.09,0.49}{##1}}}
\expandafter\def\csname PY@tok@vm\endcsname{\def\PY@tc##1{\textcolor[rgb]{0.10,0.09,0.49}{##1}}}
\expandafter\def\csname PY@tok@sa\endcsname{\def\PY@tc##1{\textcolor[rgb]{0.73,0.13,0.13}{##1}}}
\expandafter\def\csname PY@tok@sb\endcsname{\def\PY@tc##1{\textcolor[rgb]{0.73,0.13,0.13}{##1}}}
\expandafter\def\csname PY@tok@sc\endcsname{\def\PY@tc##1{\textcolor[rgb]{0.73,0.13,0.13}{##1}}}
\expandafter\def\csname PY@tok@dl\endcsname{\def\PY@tc##1{\textcolor[rgb]{0.73,0.13,0.13}{##1}}}
\expandafter\def\csname PY@tok@s2\endcsname{\def\PY@tc##1{\textcolor[rgb]{0.73,0.13,0.13}{##1}}}
\expandafter\def\csname PY@tok@sh\endcsname{\def\PY@tc##1{\textcolor[rgb]{0.73,0.13,0.13}{##1}}}
\expandafter\def\csname PY@tok@s1\endcsname{\def\PY@tc##1{\textcolor[rgb]{0.73,0.13,0.13}{##1}}}
\expandafter\def\csname PY@tok@mb\endcsname{\def\PY@tc##1{\textcolor[rgb]{0.40,0.40,0.40}{##1}}}
\expandafter\def\csname PY@tok@mf\endcsname{\def\PY@tc##1{\textcolor[rgb]{0.40,0.40,0.40}{##1}}}
\expandafter\def\csname PY@tok@mh\endcsname{\def\PY@tc##1{\textcolor[rgb]{0.40,0.40,0.40}{##1}}}
\expandafter\def\csname PY@tok@mi\endcsname{\def\PY@tc##1{\textcolor[rgb]{0.40,0.40,0.40}{##1}}}
\expandafter\def\csname PY@tok@il\endcsname{\def\PY@tc##1{\textcolor[rgb]{0.40,0.40,0.40}{##1}}}
\expandafter\def\csname PY@tok@mo\endcsname{\def\PY@tc##1{\textcolor[rgb]{0.40,0.40,0.40}{##1}}}
\expandafter\def\csname PY@tok@ch\endcsname{\let\PY@it=\textit\def\PY@tc##1{\textcolor[rgb]{0.25,0.50,0.50}{##1}}}
\expandafter\def\csname PY@tok@cm\endcsname{\let\PY@it=\textit\def\PY@tc##1{\textcolor[rgb]{0.25,0.50,0.50}{##1}}}
\expandafter\def\csname PY@tok@cpf\endcsname{\let\PY@it=\textit\def\PY@tc##1{\textcolor[rgb]{0.25,0.50,0.50}{##1}}}
\expandafter\def\csname PY@tok@c1\endcsname{\let\PY@it=\textit\def\PY@tc##1{\textcolor[rgb]{0.25,0.50,0.50}{##1}}}
\expandafter\def\csname PY@tok@cs\endcsname{\let\PY@it=\textit\def\PY@tc##1{\textcolor[rgb]{0.25,0.50,0.50}{##1}}}

\def\PYZbs{\char`\\}
\def\PYZus{\char`\_}
\def\PYZob{\char`\{}
\def\PYZcb{\char`\}}
\def\PYZca{\char`\^}
\def\PYZam{\char`\&}
\def\PYZlt{\char`\<}
\def\PYZgt{\char`\>}
\def\PYZsh{\char`\#}
\def\PYZpc{\char`\%}
\def\PYZdl{\char`\$}
\def\PYZhy{\char`\-}
\def\PYZsq{\char`\'}
\def\PYZdq{\char`\"}
\def\PYZti{\char`\~}
% for compatibility with earlier versions
\def\PYZat{@}
\def\PYZlb{[}
\def\PYZrb{]}
\makeatother


    % Exact colors from NB
    \definecolor{incolor}{rgb}{0.0, 0.0, 0.5}
    \definecolor{outcolor}{rgb}{0.545, 0.0, 0.0}



    
    % Prevent overflowing lines due to hard-to-break entities
    \sloppy 
    % Setup hyperref package
    \hypersetup{
      breaklinks=true,  % so long urls are correctly broken across lines
      colorlinks=true,
      urlcolor=urlcolor,
      linkcolor=linkcolor,
      citecolor=citecolor,
      }
    % Slightly bigger margins than the latex defaults
    
    \geometry{verbose,tmargin=1in,bmargin=1in,lmargin=1in,rmargin=1in}
    
    

    \begin{document}
    
    
    \maketitle
    
    

    
    \hypertarget{saving-and-loading-models}{%
\section{Saving and Loading Models}\label{saving-and-loading-models}}

In this notebook, I'll show you how to save and load models with
PyTorch. This is important because you'll often want to load previously
trained models to use in making predictions or to continue training on
new data.

    \begin{Verbatim}[commandchars=\\\{\}]
{\color{incolor}In [{\color{incolor}1}]:} \PY{o}{\PYZpc{}}\PY{k}{matplotlib} inline
        \PY{o}{\PYZpc{}}\PY{k}{config} InlineBackend.figure\PYZus{}format = \PYZsq{}retina\PYZsq{}
        
        \PY{k+kn}{import} \PY{n+nn}{matplotlib}\PY{n+nn}{.}\PY{n+nn}{pyplot} \PY{k}{as} \PY{n+nn}{plt}
        
        \PY{k+kn}{import} \PY{n+nn}{torch}
        \PY{k+kn}{from} \PY{n+nn}{torch} \PY{k}{import} \PY{n}{nn}
        \PY{k+kn}{from} \PY{n+nn}{torch} \PY{k}{import} \PY{n}{optim}
        \PY{k+kn}{import} \PY{n+nn}{torch}\PY{n+nn}{.}\PY{n+nn}{nn}\PY{n+nn}{.}\PY{n+nn}{functional} \PY{k}{as} \PY{n+nn}{F}
        \PY{k+kn}{from} \PY{n+nn}{torchvision} \PY{k}{import} \PY{n}{datasets}\PY{p}{,} \PY{n}{transforms}
        
        \PY{k+kn}{import} \PY{n+nn}{helper}
        \PY{k+kn}{import} \PY{n+nn}{fc\PYZus{}model}
\end{Verbatim}


    \begin{Verbatim}[commandchars=\\\{\}]
{\color{incolor}In [{\color{incolor}2}]:} \PY{c+c1}{\PYZsh{} Define a transform to normalize the data}
        \PY{n}{transform} \PY{o}{=} \PY{n}{transforms}\PY{o}{.}\PY{n}{Compose}\PY{p}{(}\PY{p}{[}\PY{n}{transforms}\PY{o}{.}\PY{n}{ToTensor}\PY{p}{(}\PY{p}{)}\PY{p}{,}
                                        \PY{n}{transforms}\PY{o}{.}\PY{n}{Normalize}\PY{p}{(}\PY{p}{(}\PY{l+m+mf}{0.5}\PY{p}{,}\PY{p}{)}\PY{p}{,} \PY{p}{(}\PY{l+m+mf}{0.5}\PY{p}{,}\PY{p}{)}\PY{p}{)}\PY{p}{]}\PY{p}{)}
        \PY{c+c1}{\PYZsh{} Download and load the training data}
        \PY{n}{trainset} \PY{o}{=} \PY{n}{datasets}\PY{o}{.}\PY{n}{FashionMNIST}\PY{p}{(}\PY{l+s+s1}{\PYZsq{}}\PY{l+s+s1}{\PYZti{}/.pytorch/F\PYZus{}MNIST\PYZus{}data/}\PY{l+s+s1}{\PYZsq{}}\PY{p}{,} \PY{n}{download}\PY{o}{=}\PY{k+kc}{True}\PY{p}{,} \PY{n}{train}\PY{o}{=}\PY{k+kc}{True}\PY{p}{,} \PY{n}{transform}\PY{o}{=}\PY{n}{transform}\PY{p}{)}
        \PY{n}{trainloader} \PY{o}{=} \PY{n}{torch}\PY{o}{.}\PY{n}{utils}\PY{o}{.}\PY{n}{data}\PY{o}{.}\PY{n}{DataLoader}\PY{p}{(}\PY{n}{trainset}\PY{p}{,} \PY{n}{batch\PYZus{}size}\PY{o}{=}\PY{l+m+mi}{64}\PY{p}{,} \PY{n}{shuffle}\PY{o}{=}\PY{k+kc}{True}\PY{p}{)}
        
        \PY{c+c1}{\PYZsh{} Download and load the test data}
        \PY{n}{testset} \PY{o}{=} \PY{n}{datasets}\PY{o}{.}\PY{n}{FashionMNIST}\PY{p}{(}\PY{l+s+s1}{\PYZsq{}}\PY{l+s+s1}{\PYZti{}/.pytorch/F\PYZus{}MNIST\PYZus{}data/}\PY{l+s+s1}{\PYZsq{}}\PY{p}{,} \PY{n}{download}\PY{o}{=}\PY{k+kc}{True}\PY{p}{,} \PY{n}{train}\PY{o}{=}\PY{k+kc}{False}\PY{p}{,} \PY{n}{transform}\PY{o}{=}\PY{n}{transform}\PY{p}{)}
        \PY{n}{testloader} \PY{o}{=} \PY{n}{torch}\PY{o}{.}\PY{n}{utils}\PY{o}{.}\PY{n}{data}\PY{o}{.}\PY{n}{DataLoader}\PY{p}{(}\PY{n}{testset}\PY{p}{,} \PY{n}{batch\PYZus{}size}\PY{o}{=}\PY{l+m+mi}{64}\PY{p}{,} \PY{n}{shuffle}\PY{o}{=}\PY{k+kc}{True}\PY{p}{)}
\end{Verbatim}


    Here we can see one of the images.

    \begin{Verbatim}[commandchars=\\\{\}]
{\color{incolor}In [{\color{incolor}3}]:} \PY{n}{image}\PY{p}{,} \PY{n}{label} \PY{o}{=} \PY{n+nb}{next}\PY{p}{(}\PY{n+nb}{iter}\PY{p}{(}\PY{n}{trainloader}\PY{p}{)}\PY{p}{)}
        \PY{n}{helper}\PY{o}{.}\PY{n}{imshow}\PY{p}{(}\PY{n}{image}\PY{p}{[}\PY{l+m+mi}{0}\PY{p}{,}\PY{p}{:}\PY{p}{]}\PY{p}{)}\PY{p}{;}
\end{Verbatim}


    \begin{center}
    \adjustimage{max size={0.9\linewidth}{0.9\paperheight}}{output_4_0.png}
    \end{center}
    { \hspace*{\fill} \\}
    
    \hypertarget{train-a-network}{%
\section{Train a network}\label{train-a-network}}

To make things more concise here, I moved the model architecture and
training code from the last part to a file called \texttt{fc\_model}.
Importing this, we can easily create a fully-connected network with
\texttt{fc\_model.Network}, and train the network using
\texttt{fc\_model.train}. I'll use this model (once it's trained) to
demonstrate how we can save and load models.

    \begin{Verbatim}[commandchars=\\\{\}]
{\color{incolor}In [{\color{incolor}4}]:} \PY{c+c1}{\PYZsh{} Create the network, define the criterion and optimizer}
        
        \PY{n}{model} \PY{o}{=} \PY{n}{fc\PYZus{}model}\PY{o}{.}\PY{n}{Network}\PY{p}{(}\PY{l+m+mi}{784}\PY{p}{,} \PY{l+m+mi}{10}\PY{p}{,} \PY{p}{[}\PY{l+m+mi}{512}\PY{p}{,} \PY{l+m+mi}{256}\PY{p}{,} \PY{l+m+mi}{128}\PY{p}{]}\PY{p}{)}
        \PY{n}{criterion} \PY{o}{=} \PY{n}{nn}\PY{o}{.}\PY{n}{NLLLoss}\PY{p}{(}\PY{p}{)}
        \PY{n}{optimizer} \PY{o}{=} \PY{n}{optim}\PY{o}{.}\PY{n}{Adam}\PY{p}{(}\PY{n}{model}\PY{o}{.}\PY{n}{parameters}\PY{p}{(}\PY{p}{)}\PY{p}{,} \PY{n}{lr}\PY{o}{=}\PY{l+m+mf}{0.001}\PY{p}{)}
\end{Verbatim}


    \begin{Verbatim}[commandchars=\\\{\}]
{\color{incolor}In [{\color{incolor}5}]:} \PY{n}{fc\PYZus{}model}\PY{o}{.}\PY{n}{train}\PY{p}{(}\PY{n}{model}\PY{p}{,} \PY{n}{trainloader}\PY{p}{,} \PY{n}{testloader}\PY{p}{,} \PY{n}{criterion}\PY{p}{,} \PY{n}{optimizer}\PY{p}{,} \PY{n}{epochs}\PY{o}{=}\PY{l+m+mi}{2}\PY{p}{)}
\end{Verbatim}


    \begin{Verbatim}[commandchars=\\\{\}]
/opt/conda/lib/python3.6/site-packages/torch/autograd/\_\_init\_\_.py:132: UserWarning: CUDA initialization: Found no NVIDIA driver on your system. Please check that you have an NVIDIA GPU and installed a driver from http://www.nvidia.com/Download/index.aspx (Triggered internally at  /pytorch/c10/cuda/CUDAFunctions.cpp:100.)
  allow\_unreachable=True)  \# allow\_unreachable flag

    \end{Verbatim}

    \begin{Verbatim}[commandchars=\\\{\}]
Epoch: 1/2..  Training Loss: 1.731..  Test Loss: 0.941..  Test Accuracy: 0.668
Epoch: 1/2..  Training Loss: 1.007..  Test Loss: 0.731..  Test Accuracy: 0.723
Epoch: 1/2..  Training Loss: 0.863..  Test Loss: 0.669..  Test Accuracy: 0.749
Epoch: 1/2..  Training Loss: 0.796..  Test Loss: 0.676..  Test Accuracy: 0.733
Epoch: 1/2..  Training Loss: 0.776..  Test Loss: 0.605..  Test Accuracy: 0.771
Epoch: 1/2..  Training Loss: 0.687..  Test Loss: 0.595..  Test Accuracy: 0.770
Epoch: 1/2..  Training Loss: 0.687..  Test Loss: 0.576..  Test Accuracy: 0.787
Epoch: 1/2..  Training Loss: 0.713..  Test Loss: 0.597..  Test Accuracy: 0.777
Epoch: 1/2..  Training Loss: 0.677..  Test Loss: 0.566..  Test Accuracy: 0.788
Epoch: 1/2..  Training Loss: 0.647..  Test Loss: 0.554..  Test Accuracy: 0.793
Epoch: 1/2..  Training Loss: 0.624..  Test Loss: 0.563..  Test Accuracy: 0.784
Epoch: 1/2..  Training Loss: 0.628..  Test Loss: 0.522..  Test Accuracy: 0.801
Epoch: 1/2..  Training Loss: 0.610..  Test Loss: 0.508..  Test Accuracy: 0.809
Epoch: 1/2..  Training Loss: 0.595..  Test Loss: 0.510..  Test Accuracy: 0.816
Epoch: 1/2..  Training Loss: 0.553..  Test Loss: 0.538..  Test Accuracy: 0.797
Epoch: 1/2..  Training Loss: 0.596..  Test Loss: 0.513..  Test Accuracy: 0.815
Epoch: 1/2..  Training Loss: 0.615..  Test Loss: 0.505..  Test Accuracy: 0.816
Epoch: 1/2..  Training Loss: 0.544..  Test Loss: 0.484..  Test Accuracy: 0.821
Epoch: 1/2..  Training Loss: 0.575..  Test Loss: 0.494..  Test Accuracy: 0.822
Epoch: 1/2..  Training Loss: 0.626..  Test Loss: 0.485..  Test Accuracy: 0.827
Epoch: 1/2..  Training Loss: 0.572..  Test Loss: 0.479..  Test Accuracy: 0.824
Epoch: 1/2..  Training Loss: 0.571..  Test Loss: 0.492..  Test Accuracy: 0.818
Epoch: 1/2..  Training Loss: 0.566..  Test Loss: 0.484..  Test Accuracy: 0.823
Epoch: 2/2..  Training Loss: 0.560..  Test Loss: 0.485..  Test Accuracy: 0.819
Epoch: 2/2..  Training Loss: 0.550..  Test Loss: 0.499..  Test Accuracy: 0.819
Epoch: 2/2..  Training Loss: 0.569..  Test Loss: 0.490..  Test Accuracy: 0.820
Epoch: 2/2..  Training Loss: 0.531..  Test Loss: 0.475..  Test Accuracy: 0.828
Epoch: 2/2..  Training Loss: 0.573..  Test Loss: 0.473..  Test Accuracy: 0.828
Epoch: 2/2..  Training Loss: 0.520..  Test Loss: 0.491..  Test Accuracy: 0.828
Epoch: 2/2..  Training Loss: 0.520..  Test Loss: 0.483..  Test Accuracy: 0.825
Epoch: 2/2..  Training Loss: 0.510..  Test Loss: 0.462..  Test Accuracy: 0.829
Epoch: 2/2..  Training Loss: 0.530..  Test Loss: 0.484..  Test Accuracy: 0.819
Epoch: 2/2..  Training Loss: 0.523..  Test Loss: 0.458..  Test Accuracy: 0.831
Epoch: 2/2..  Training Loss: 0.544..  Test Loss: 0.467..  Test Accuracy: 0.825
Epoch: 2/2..  Training Loss: 0.536..  Test Loss: 0.458..  Test Accuracy: 0.833
Epoch: 2/2..  Training Loss: 0.531..  Test Loss: 0.463..  Test Accuracy: 0.825
Epoch: 2/2..  Training Loss: 0.521..  Test Loss: 0.450..  Test Accuracy: 0.836
Epoch: 2/2..  Training Loss: 0.543..  Test Loss: 0.458..  Test Accuracy: 0.835
Epoch: 2/2..  Training Loss: 0.529..  Test Loss: 0.470..  Test Accuracy: 0.830
Epoch: 2/2..  Training Loss: 0.480..  Test Loss: 0.463..  Test Accuracy: 0.833
Epoch: 2/2..  Training Loss: 0.516..  Test Loss: 0.451..  Test Accuracy: 0.837
Epoch: 2/2..  Training Loss: 0.504..  Test Loss: 0.452..  Test Accuracy: 0.833
Epoch: 2/2..  Training Loss: 0.515..  Test Loss: 0.453..  Test Accuracy: 0.834
Epoch: 2/2..  Training Loss: 0.515..  Test Loss: 0.447..  Test Accuracy: 0.838
Epoch: 2/2..  Training Loss: 0.530..  Test Loss: 0.444..  Test Accuracy: 0.844
Epoch: 2/2..  Training Loss: 0.487..  Test Loss: 0.446..  Test Accuracy: 0.837

    \end{Verbatim}

    \hypertarget{saving-and-loading-networks}{%
\subsection{Saving and loading
networks}\label{saving-and-loading-networks}}

As you can imagine, it's impractical to train a network every time you
need to use it. Instead, we can save trained networks then load them
later to train more or use them for predictions.

The parameters for PyTorch networks are stored in a model's
\texttt{state\_dict}. We can see the state dict contains the weight and
bias matrices for each of our layers.

    \begin{Verbatim}[commandchars=\\\{\}]
{\color{incolor}In [{\color{incolor}6}]:} \PY{n+nb}{print}\PY{p}{(}\PY{l+s+s2}{\PYZdq{}}\PY{l+s+s2}{Our model: }\PY{l+s+se}{\PYZbs{}n}\PY{l+s+se}{\PYZbs{}n}\PY{l+s+s2}{\PYZdq{}}\PY{p}{,} \PY{n}{model}\PY{p}{,} \PY{l+s+s1}{\PYZsq{}}\PY{l+s+se}{\PYZbs{}n}\PY{l+s+s1}{\PYZsq{}}\PY{p}{)}
        \PY{n+nb}{print}\PY{p}{(}\PY{l+s+s2}{\PYZdq{}}\PY{l+s+s2}{The state dict keys: }\PY{l+s+se}{\PYZbs{}n}\PY{l+s+se}{\PYZbs{}n}\PY{l+s+s2}{\PYZdq{}}\PY{p}{,} \PY{n}{model}\PY{o}{.}\PY{n}{state\PYZus{}dict}\PY{p}{(}\PY{p}{)}\PY{o}{.}\PY{n}{keys}\PY{p}{(}\PY{p}{)}\PY{p}{)}
\end{Verbatim}


    \begin{Verbatim}[commandchars=\\\{\}]
Our model: 

 Network(
  (hidden\_layers): ModuleList(
    (0): Linear(in\_features=784, out\_features=512, bias=True)
    (1): Linear(in\_features=512, out\_features=256, bias=True)
    (2): Linear(in\_features=256, out\_features=128, bias=True)
  )
  (output): Linear(in\_features=128, out\_features=10, bias=True)
  (dropout): Dropout(p=0.5, inplace=False)
) 

The state dict keys: 

 odict\_keys(['hidden\_layers.0.weight', 'hidden\_layers.0.bias', 'hidden\_layers.1.weight', 'hidden\_layers.1.bias', 'hidden\_layers.2.weight', 'hidden\_layers.2.bias', 'output.weight', 'output.bias'])

    \end{Verbatim}

    The simplest thing to do is simply save the state dict with
\texttt{torch.save}. For example, we can save it to a file
\texttt{\textquotesingle{}network\_A.pth\textquotesingle{}}.

    \begin{Verbatim}[commandchars=\\\{\}]
{\color{incolor}In [{\color{incolor}7}]:} \PY{n}{torch}\PY{o}{.}\PY{n}{save}\PY{p}{(}\PY{n}{model}\PY{o}{.}\PY{n}{state\PYZus{}dict}\PY{p}{(}\PY{p}{)}\PY{p}{,} \PY{l+s+s1}{\PYZsq{}}\PY{l+s+s1}{network\PYZus{}A.pth}\PY{l+s+s1}{\PYZsq{}}\PY{p}{)}
\end{Verbatim}


    Then we can load the state dict with \texttt{torch.load}.

    \begin{Verbatim}[commandchars=\\\{\}]
{\color{incolor}In [{\color{incolor}8}]:} \PY{n}{state\PYZus{}dict} \PY{o}{=} \PY{n}{torch}\PY{o}{.}\PY{n}{load}\PY{p}{(}\PY{l+s+s1}{\PYZsq{}}\PY{l+s+s1}{network\PYZus{}A.pth}\PY{l+s+s1}{\PYZsq{}}\PY{p}{)}
        \PY{n+nb}{print}\PY{p}{(}\PY{n}{state\PYZus{}dict}\PY{o}{.}\PY{n}{keys}\PY{p}{(}\PY{p}{)}\PY{p}{)}
\end{Verbatim}


    \begin{Verbatim}[commandchars=\\\{\}]
odict\_keys(['hidden\_layers.0.weight', 'hidden\_layers.0.bias', 'hidden\_layers.1.weight', 'hidden\_layers.1.bias', 'hidden\_layers.2.weight', 'hidden\_layers.2.bias', 'output.weight', 'output.bias'])

    \end{Verbatim}

    And to load the state dict in to the network, you do
\texttt{model.load\_state\_dict(state\_dict)}.

    \begin{Verbatim}[commandchars=\\\{\}]
{\color{incolor}In [{\color{incolor}9}]:} \PY{n}{model}\PY{o}{.}\PY{n}{load\PYZus{}state\PYZus{}dict}\PY{p}{(}\PY{n}{state\PYZus{}dict}\PY{p}{)}
\end{Verbatim}


\begin{Verbatim}[commandchars=\\\{\}]
{\color{outcolor}Out[{\color{outcolor}9}]:} <All keys matched successfully>
\end{Verbatim}
            
    Seems pretty straightforward, but as usual it's a bit more complicated.
Loading the state dict works only if the model architecture is exactly
the same as the checkpoint architecture. If I create a model with a
different architecture, this fails.

    \begin{Verbatim}[commandchars=\\\{\}]
{\color{incolor}In [{\color{incolor}10}]:} \PY{c+c1}{\PYZsh{} Try this}
         \PY{n}{model} \PY{o}{=} \PY{n}{fc\PYZus{}model}\PY{o}{.}\PY{n}{Network}\PY{p}{(}\PY{l+m+mi}{784}\PY{p}{,} \PY{l+m+mi}{10}\PY{p}{,} \PY{p}{[}\PY{l+m+mi}{400}\PY{p}{,} \PY{l+m+mi}{200}\PY{p}{,} \PY{l+m+mi}{100}\PY{p}{]}\PY{p}{)}
         \PY{c+c1}{\PYZsh{} This will throw an error because the tensor sizes are wrong!}
         \PY{n}{model}\PY{o}{.}\PY{n}{load\PYZus{}state\PYZus{}dict}\PY{p}{(}\PY{n}{state\PYZus{}dict}\PY{p}{)}
\end{Verbatim}


    \begin{Verbatim}[commandchars=\\\{\}]

        ---------------------------------------------------------------------------

        RuntimeError                              Traceback (most recent call last)

        <ipython-input-10-d859c59ebec0> in <module>
          2 model = fc\_model.Network(784, 10, [400, 200, 100])
          3 \# This will throw an error because the tensor sizes are wrong!
    ----> 4 model.load\_state\_dict(state\_dict)
    

        /opt/conda/lib/python3.6/site-packages/torch/nn/modules/module.py in load\_state\_dict(self, state\_dict, strict)
       1050         if len(error\_msgs) > 0:
       1051             raise RuntimeError('Error(s) in loading state\_dict for \{\}:\textbackslash{}n\textbackslash{}t\{\}'.format(
    -> 1052                                self.\_\_class\_\_.\_\_name\_\_, "\textbackslash{}n\textbackslash{}t".join(error\_msgs)))
       1053         return \_IncompatibleKeys(missing\_keys, unexpected\_keys)
       1054 


        RuntimeError: Error(s) in loading state\_dict for Network:
    	size mismatch for hidden\_layers.0.weight: copying a param with shape torch.Size([512, 784]) from checkpoint, the shape in current model is torch.Size([400, 784]).
    	size mismatch for hidden\_layers.0.bias: copying a param with shape torch.Size([512]) from checkpoint, the shape in current model is torch.Size([400]).
    	size mismatch for hidden\_layers.1.weight: copying a param with shape torch.Size([256, 512]) from checkpoint, the shape in current model is torch.Size([200, 400]).
    	size mismatch for hidden\_layers.1.bias: copying a param with shape torch.Size([256]) from checkpoint, the shape in current model is torch.Size([200]).
    	size mismatch for hidden\_layers.2.weight: copying a param with shape torch.Size([128, 256]) from checkpoint, the shape in current model is torch.Size([100, 200]).
    	size mismatch for hidden\_layers.2.bias: copying a param with shape torch.Size([128]) from checkpoint, the shape in current model is torch.Size([100]).
    	size mismatch for output.weight: copying a param with shape torch.Size([10, 128]) from checkpoint, the shape in current model is torch.Size([10, 100]).

    \end{Verbatim}

    This means we need to rebuild the model exactly as it was when trained.
Information about the model architecture needs to be saved in the
checkpoint, along with the state dict. To do this, you build a
dictionary with all the information you need to compeletely rebuild the
model.

    \begin{Verbatim}[commandchars=\\\{\}]
{\color{incolor}In [{\color{incolor}11}]:} \PY{n}{checkpoint} \PY{o}{=} \PY{p}{\PYZob{}}\PY{l+s+s1}{\PYZsq{}}\PY{l+s+s1}{input\PYZus{}size}\PY{l+s+s1}{\PYZsq{}}\PY{p}{:} \PY{l+m+mi}{784}\PY{p}{,}
                       \PY{l+s+s1}{\PYZsq{}}\PY{l+s+s1}{output\PYZus{}size}\PY{l+s+s1}{\PYZsq{}}\PY{p}{:} \PY{l+m+mi}{10}\PY{p}{,}
                       \PY{l+s+s1}{\PYZsq{}}\PY{l+s+s1}{hidden\PYZus{}layers}\PY{l+s+s1}{\PYZsq{}}\PY{p}{:} \PY{p}{[}\PY{n}{each}\PY{o}{.}\PY{n}{out\PYZus{}features} \PY{k}{for} \PY{n}{each} \PY{o+ow}{in} \PY{n}{model}\PY{o}{.}\PY{n}{hidden\PYZus{}layers}\PY{p}{]}\PY{p}{,}
                       \PY{l+s+s1}{\PYZsq{}}\PY{l+s+s1}{state\PYZus{}dict}\PY{l+s+s1}{\PYZsq{}}\PY{p}{:} \PY{n}{model}\PY{o}{.}\PY{n}{state\PYZus{}dict}\PY{p}{(}\PY{p}{)}\PY{p}{\PYZcb{}}
         
         \PY{n}{torch}\PY{o}{.}\PY{n}{save}\PY{p}{(}\PY{n}{checkpoint}\PY{p}{,} \PY{l+s+s1}{\PYZsq{}}\PY{l+s+s1}{network\PYZus{}B.pth}\PY{l+s+s1}{\PYZsq{}}\PY{p}{)}
\end{Verbatim}


    Now the checkpoint has all the necessary information to rebuild the
trained model. You can easily make that a function if you want.
Similarly, we can write a function to load checkpoints.

    \begin{Verbatim}[commandchars=\\\{\}]
{\color{incolor}In [{\color{incolor}12}]:} \PY{k}{def} \PY{n+nf}{load\PYZus{}checkpoint}\PY{p}{(}\PY{n}{filepath}\PY{p}{)}\PY{p}{:}
             \PY{n}{checkpoint} \PY{o}{=} \PY{n}{torch}\PY{o}{.}\PY{n}{load}\PY{p}{(}\PY{n}{filepath}\PY{p}{)}
             \PY{n}{model} \PY{o}{=} \PY{n}{fc\PYZus{}model}\PY{o}{.}\PY{n}{Network}\PY{p}{(}\PY{n}{checkpoint}\PY{p}{[}\PY{l+s+s1}{\PYZsq{}}\PY{l+s+s1}{input\PYZus{}size}\PY{l+s+s1}{\PYZsq{}}\PY{p}{]}\PY{p}{,}
                                      \PY{n}{checkpoint}\PY{p}{[}\PY{l+s+s1}{\PYZsq{}}\PY{l+s+s1}{output\PYZus{}size}\PY{l+s+s1}{\PYZsq{}}\PY{p}{]}\PY{p}{,}
                                      \PY{n}{checkpoint}\PY{p}{[}\PY{l+s+s1}{\PYZsq{}}\PY{l+s+s1}{hidden\PYZus{}layers}\PY{l+s+s1}{\PYZsq{}}\PY{p}{]}\PY{p}{)}
             \PY{n}{model}\PY{o}{.}\PY{n}{load\PYZus{}state\PYZus{}dict}\PY{p}{(}\PY{n}{checkpoint}\PY{p}{[}\PY{l+s+s1}{\PYZsq{}}\PY{l+s+s1}{state\PYZus{}dict}\PY{l+s+s1}{\PYZsq{}}\PY{p}{]}\PY{p}{)}
             
             \PY{k}{return} \PY{n}{model}
\end{Verbatim}


    \begin{Verbatim}[commandchars=\\\{\}]
{\color{incolor}In [{\color{incolor}17}]:} \PY{c+c1}{\PYZsh{}\PYZsh{} Exercise 6.1: }
         \PY{c+c1}{\PYZsh{}\PYZsh{} TODO: Import saved network B, loads it to network and print network model}
         
         \PY{n}{networkB} \PY{o}{=} \PY{n}{load\PYZus{}checkpoint}\PY{p}{(}\PY{l+s+s1}{\PYZsq{}}\PY{l+s+s1}{network\PYZus{}B.pth}\PY{l+s+s1}{\PYZsq{}}\PY{p}{)}
         \PY{n+nb}{print}\PY{p}{(}\PY{n}{networkB}\PY{p}{)}
\end{Verbatim}


    \begin{Verbatim}[commandchars=\\\{\}]
Network(
  (hidden\_layers): ModuleList(
    (0): Linear(in\_features=784, out\_features=400, bias=True)
    (1): Linear(in\_features=400, out\_features=200, bias=True)
    (2): Linear(in\_features=200, out\_features=100, bias=True)
  )
  (output): Linear(in\_features=100, out\_features=10, bias=True)
  (dropout): Dropout(p=0.5, inplace=False)
)

    \end{Verbatim}

    \hypertarget{reflection}{%
\subsection{Reflection}\label{reflection}}

Answer briefly following questions (in English or Finnish): - What is
ONNX?

    \begin{itemize}
\tightlist
\item
  Open Neural Network Exchange (ONNX) on open source AI kehitys
  ekosysteemi, joka tarjoaa kone- ja syväoppimismalleja.
\end{itemize}


    % Add a bibliography block to the postdoc
    
    
    
    \end{document}
